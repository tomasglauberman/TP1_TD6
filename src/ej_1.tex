% Options for packages loaded elsewhere
\PassOptionsToPackage{unicode}{hyperref}
\PassOptionsToPackage{hyphens}{url}
%
\documentclass[
]{article}
\usepackage{amsmath,amssymb}
\usepackage{iftex}
\ifPDFTeX
  \usepackage[T1]{fontenc}
  \usepackage[utf8]{inputenc}
  \usepackage{textcomp} % provide euro and other symbols
\else % if luatex or xetex
  \usepackage{unicode-math} % this also loads fontspec
  \defaultfontfeatures{Scale=MatchLowercase}
  \defaultfontfeatures[\rmfamily]{Ligatures=TeX,Scale=1}
\fi
\usepackage{lmodern}
\ifPDFTeX\else
  % xetex/luatex font selection
\fi
% Use upquote if available, for straight quotes in verbatim environments
\IfFileExists{upquote.sty}{\usepackage{upquote}}{}
\IfFileExists{microtype.sty}{% use microtype if available
  \usepackage[]{microtype}
  \UseMicrotypeSet[protrusion]{basicmath} % disable protrusion for tt fonts
}{}
\makeatletter
\@ifundefined{KOMAClassName}{% if non-KOMA class
  \IfFileExists{parskip.sty}{%
    \usepackage{parskip}
  }{% else
    \setlength{\parindent}{0pt}
    \setlength{\parskip}{6pt plus 2pt minus 1pt}}
}{% if KOMA class
  \KOMAoptions{parskip=half}}
\makeatother
\usepackage{xcolor}
\usepackage[margin=1in]{geometry}
\usepackage{color}
\usepackage{fancyvrb}
\newcommand{\VerbBar}{|}
\newcommand{\VERB}{\Verb[commandchars=\\\{\}]}
\DefineVerbatimEnvironment{Highlighting}{Verbatim}{commandchars=\\\{\}}
% Add ',fontsize=\small' for more characters per line
\usepackage{framed}
\definecolor{shadecolor}{RGB}{248,248,248}
\newenvironment{Shaded}{\begin{snugshade}}{\end{snugshade}}
\newcommand{\AlertTok}[1]{\textcolor[rgb]{0.94,0.16,0.16}{#1}}
\newcommand{\AnnotationTok}[1]{\textcolor[rgb]{0.56,0.35,0.01}{\textbf{\textit{#1}}}}
\newcommand{\AttributeTok}[1]{\textcolor[rgb]{0.13,0.29,0.53}{#1}}
\newcommand{\BaseNTok}[1]{\textcolor[rgb]{0.00,0.00,0.81}{#1}}
\newcommand{\BuiltInTok}[1]{#1}
\newcommand{\CharTok}[1]{\textcolor[rgb]{0.31,0.60,0.02}{#1}}
\newcommand{\CommentTok}[1]{\textcolor[rgb]{0.56,0.35,0.01}{\textit{#1}}}
\newcommand{\CommentVarTok}[1]{\textcolor[rgb]{0.56,0.35,0.01}{\textbf{\textit{#1}}}}
\newcommand{\ConstantTok}[1]{\textcolor[rgb]{0.56,0.35,0.01}{#1}}
\newcommand{\ControlFlowTok}[1]{\textcolor[rgb]{0.13,0.29,0.53}{\textbf{#1}}}
\newcommand{\DataTypeTok}[1]{\textcolor[rgb]{0.13,0.29,0.53}{#1}}
\newcommand{\DecValTok}[1]{\textcolor[rgb]{0.00,0.00,0.81}{#1}}
\newcommand{\DocumentationTok}[1]{\textcolor[rgb]{0.56,0.35,0.01}{\textbf{\textit{#1}}}}
\newcommand{\ErrorTok}[1]{\textcolor[rgb]{0.64,0.00,0.00}{\textbf{#1}}}
\newcommand{\ExtensionTok}[1]{#1}
\newcommand{\FloatTok}[1]{\textcolor[rgb]{0.00,0.00,0.81}{#1}}
\newcommand{\FunctionTok}[1]{\textcolor[rgb]{0.13,0.29,0.53}{\textbf{#1}}}
\newcommand{\ImportTok}[1]{#1}
\newcommand{\InformationTok}[1]{\textcolor[rgb]{0.56,0.35,0.01}{\textbf{\textit{#1}}}}
\newcommand{\KeywordTok}[1]{\textcolor[rgb]{0.13,0.29,0.53}{\textbf{#1}}}
\newcommand{\NormalTok}[1]{#1}
\newcommand{\OperatorTok}[1]{\textcolor[rgb]{0.81,0.36,0.00}{\textbf{#1}}}
\newcommand{\OtherTok}[1]{\textcolor[rgb]{0.56,0.35,0.01}{#1}}
\newcommand{\PreprocessorTok}[1]{\textcolor[rgb]{0.56,0.35,0.01}{\textit{#1}}}
\newcommand{\RegionMarkerTok}[1]{#1}
\newcommand{\SpecialCharTok}[1]{\textcolor[rgb]{0.81,0.36,0.00}{\textbf{#1}}}
\newcommand{\SpecialStringTok}[1]{\textcolor[rgb]{0.31,0.60,0.02}{#1}}
\newcommand{\StringTok}[1]{\textcolor[rgb]{0.31,0.60,0.02}{#1}}
\newcommand{\VariableTok}[1]{\textcolor[rgb]{0.00,0.00,0.00}{#1}}
\newcommand{\VerbatimStringTok}[1]{\textcolor[rgb]{0.31,0.60,0.02}{#1}}
\newcommand{\WarningTok}[1]{\textcolor[rgb]{0.56,0.35,0.01}{\textbf{\textit{#1}}}}
\usepackage{graphicx}
\makeatletter
\def\maxwidth{\ifdim\Gin@nat@width>\linewidth\linewidth\else\Gin@nat@width\fi}
\def\maxheight{\ifdim\Gin@nat@height>\textheight\textheight\else\Gin@nat@height\fi}
\makeatother
% Scale images if necessary, so that they will not overflow the page
% margins by default, and it is still possible to overwrite the defaults
% using explicit options in \includegraphics[width, height, ...]{}
\setkeys{Gin}{width=\maxwidth,height=\maxheight,keepaspectratio}
% Set default figure placement to htbp
\makeatletter
\def\fps@figure{htbp}
\makeatother
\setlength{\emergencystretch}{3em} % prevent overfull lines
\providecommand{\tightlist}{%
  \setlength{\itemsep}{0pt}\setlength{\parskip}{0pt}}
\setcounter{secnumdepth}{-\maxdimen} % remove section numbering
\ifLuaTeX
  \usepackage{selnolig}  % disable illegal ligatures
\fi
\IfFileExists{bookmark.sty}{\usepackage{bookmark}}{\usepackage{hyperref}}
\IfFileExists{xurl.sty}{\usepackage{xurl}}{} % add URL line breaks if available
\urlstyle{same}
\hypersetup{
  pdftitle={Dataset},
  hidelinks,
  pdfcreator={LaTeX via pandoc}}

\title{Dataset}
\author{}
\date{\vspace{-2.5em}}

\begin{document}
\maketitle

\hypertarget{ejercicio-1}{%
\section{Ejercicio 1}\label{ejercicio-1}}

\hypertarget{selecciuxf3n-de-un-conjunto-de-datos-adicional}{%
\paragraph{Selección de un conjunto de datos
adicional}\label{selecciuxf3n-de-un-conjunto-de-datos-adicional}}

El dataset seleccionado para realizar los experimentos contiene datos
relevantes para el estudio del sueño. Los datos se obtuvieron a través
de Kaggle, en el siguiente
\href{https://www.kaggle.com/datasets/uom190346a/sleep-health-and-lifestyle-dataset}{link}.
El dataset consiste de 374 observaciones con 13 variables:

\begin{itemize}
\item
  Numéricas:

  \begin{itemize}
  \item
    Age
  \item
    Sleep duration (hrs)
  \item
    Quality of sleep (1-10)
  \item
    Physical activity level (min/día)
  \item
    Stress level (1-10)
  \item
    Blood pressure
  \item
    Heart rate
  \end{itemize}
\item
  Categóricas

  \begin{itemize}
  \item
    Gender
  \item
    Occupation
  \item
    BMI Category
  \item
    Sleep disorder
  \end{itemize}
\end{itemize}

El objetivo es construir un modelo para predecir si una persona posee
algún sleep disorder en base a variables de salud, ocupación, duración y
calidad de sueño, etc.

Se debe transformar la columna de ``Blood pressure'' que viene dada en
el formato yyy/xx a 2 columnas diferentes una para cada valor: yyy, xx.
Por otro lado, también modificamos la variable a predecir, ``Sleep
Disorder'', dado que la original viene con los valores None o el nombre
del disorder, y la tranformamos en etiquetas binarias: posee o no algún
sleep disorder.

\begin{Shaded}
\begin{Highlighting}[]
\NormalTok{sleep\_data }\OtherTok{\textless{}{-}} \FunctionTok{read.csv}\NormalTok{(}\StringTok{"data/sleep\_health.csv"}\NormalTok{)}
\end{Highlighting}
\end{Shaded}

\begin{Shaded}
\begin{Highlighting}[]
\NormalTok{sleep\_data}\SpecialCharTok{$}\NormalTok{Sleep.Disorder[sleep\_data}\SpecialCharTok{$}\NormalTok{Sleep.Disorder }\SpecialCharTok{!=} \StringTok{"None"}\NormalTok{] }\OtherTok{\textless{}{-}} \StringTok{"Yes"}
\NormalTok{sleep\_data}\SpecialCharTok{$}\NormalTok{Sleep.Disorder[sleep\_data}\SpecialCharTok{$}\NormalTok{Sleep.Disorder }\SpecialCharTok{==} \StringTok{"None"}\NormalTok{] }\OtherTok{\textless{}{-}} \StringTok{"No"}
\end{Highlighting}
\end{Shaded}

\begin{Shaded}
\begin{Highlighting}[]
\CommentTok{\# Cast column and split meassurements}
\NormalTok{sleep\_data}\SpecialCharTok{$}\NormalTok{Blood.Pressure }\OtherTok{\textless{}{-}} \FunctionTok{as.character}\NormalTok{(sleep\_data}\SpecialCharTok{$}\NormalTok{Blood.Pressure)}
\NormalTok{split\_bp }\OtherTok{\textless{}{-}} \FunctionTok{strsplit}\NormalTok{(sleep\_data}\SpecialCharTok{$}\NormalTok{Blood.Pressure, }\StringTok{"/"}\NormalTok{)}

\CommentTok{\# Create new columns for systolic and diastolic}
\NormalTok{sleep\_data}\SpecialCharTok{$}\NormalTok{systolic }\OtherTok{\textless{}{-}} \FunctionTok{as.numeric}\NormalTok{(}\FunctionTok{sapply}\NormalTok{(split\_bp, }\StringTok{\textasciigrave{}}\AttributeTok{[}\StringTok{\textasciigrave{}}\NormalTok{, }\DecValTok{1}\NormalTok{))}
\NormalTok{sleep\_data}\SpecialCharTok{$}\NormalTok{diastolic }\OtherTok{\textless{}{-}} \FunctionTok{as.numeric}\NormalTok{(}\FunctionTok{sapply}\NormalTok{(split\_bp, }\StringTok{\textasciigrave{}}\AttributeTok{[}\StringTok{\textasciigrave{}}\NormalTok{, }\DecValTok{2}\NormalTok{))}

\CommentTok{\# Remove the original Blood Pressure column if desired}
\NormalTok{sleep\_data}\SpecialCharTok{$}\NormalTok{Blood.Pressure }\OtherTok{\textless{}{-}} \ConstantTok{NULL}
\end{Highlighting}
\end{Shaded}

\begin{Shaded}
\begin{Highlighting}[]
\CommentTok{\# Write the modified DataFrame to a CSV file with custom options}
\FunctionTok{write.table}\NormalTok{(sleep\_data, }\StringTok{"data/sleep\_health\_proc.csv"}\NormalTok{, }\AttributeTok{sep =} \StringTok{","}\NormalTok{, }\AttributeTok{col.names =} \ConstantTok{TRUE}\NormalTok{, }\AttributeTok{row.names =} \ConstantTok{FALSE}\NormalTok{)}
\end{Highlighting}
\end{Shaded}

\begin{Shaded}
\begin{Highlighting}[]
\NormalTok{sleep\_data }\OtherTok{\textless{}{-}} \FunctionTok{read.csv}\NormalTok{(}\StringTok{"./data/sleep\_health\_proc.csv"}\NormalTok{, }\AttributeTok{sep =} \StringTok{","}\NormalTok{)}
\FunctionTok{head}\NormalTok{(sleep\_data, }\AttributeTok{n=}\DecValTok{5}\NormalTok{)}
\end{Highlighting}
\end{Shaded}

\begin{verbatim}
##   Person.ID Gender Age           Occupation Sleep.Duration Quality.of.Sleep
## 1         1   Male  27    Software Engineer            6.1                6
## 2         2   Male  28               Doctor            6.2                6
## 3         3   Male  28               Doctor            6.2                6
## 4         4   Male  28 Sales Representative            5.9                4
## 5         5   Male  28 Sales Representative            5.9                4
##   Physical.Activity.Level Stress.Level BMI.Category Heart.Rate Daily.Steps
## 1                      42            6   Overweight         77        4200
## 2                      60            8       Normal         75       10000
## 3                      60            8       Normal         75       10000
## 4                      30            8        Obese         85        3000
## 5                      30            8        Obese         85        3000
##   Sleep.Disorder systolic diastolic
## 1             No      126        83
## 2             No      125        80
## 3             No      125        80
## 4            Yes      140        90
## 5            Yes      140        90
\end{verbatim}

\end{document}
